\documentclass[]{article}
\usepackage{lmodern}
\usepackage{amssymb,amsmath}
\usepackage{ifxetex,ifluatex}
\usepackage{fixltx2e} % provides \textsubscript
\ifnum 0\ifxetex 1\fi\ifluatex 1\fi=0 % if pdftex
  \usepackage[T1]{fontenc}
  \usepackage[utf8]{inputenc}
\else % if luatex or xelatex
  \ifxetex
    \usepackage{mathspec}
  \else
    \usepackage{fontspec}
  \fi
  \defaultfontfeatures{Ligatures=TeX,Scale=MatchLowercase}
\fi
% use upquote if available, for straight quotes in verbatim environments
\IfFileExists{upquote.sty}{\usepackage{upquote}}{}
% use microtype if available
\IfFileExists{microtype.sty}{%
\usepackage{microtype}
\UseMicrotypeSet[protrusion]{basicmath} % disable protrusion for tt fonts
}{}
\usepackage[margin=1in]{geometry}
\usepackage{hyperref}
\hypersetup{unicode=true,
            pdftitle={Georeferencing in Q},
            pdfauthor={Sam Levin},
            pdfborder={0 0 0},
            breaklinks=true}
\urlstyle{same}  % don't use monospace font for urls
\usepackage{graphicx,grffile}
\makeatletter
\def\maxwidth{\ifdim\Gin@nat@width>\linewidth\linewidth\else\Gin@nat@width\fi}
\def\maxheight{\ifdim\Gin@nat@height>\textheight\textheight\else\Gin@nat@height\fi}
\makeatother
% Scale images if necessary, so that they will not overflow the page
% margins by default, and it is still possible to overwrite the defaults
% using explicit options in \includegraphics[width, height, ...]{}
\setkeys{Gin}{width=\maxwidth,height=\maxheight,keepaspectratio}
\IfFileExists{parskip.sty}{%
\usepackage{parskip}
}{% else
\setlength{\parindent}{0pt}
\setlength{\parskip}{6pt plus 2pt minus 1pt}
}
\setlength{\emergencystretch}{3em}  % prevent overfull lines
\providecommand{\tightlist}{%
  \setlength{\itemsep}{0pt}\setlength{\parskip}{0pt}}
\setcounter{secnumdepth}{0}
% Redefines (sub)paragraphs to behave more like sections
\ifx\paragraph\undefined\else
\let\oldparagraph\paragraph
\renewcommand{\paragraph}[1]{\oldparagraph{#1}\mbox{}}
\fi
\ifx\subparagraph\undefined\else
\let\oldsubparagraph\subparagraph
\renewcommand{\subparagraph}[1]{\oldsubparagraph{#1}\mbox{}}
\fi

%%% Use protect on footnotes to avoid problems with footnotes in titles
\let\rmarkdownfootnote\footnote%
\def\footnote{\protect\rmarkdownfootnote}

%%% Change title format to be more compact
\usepackage{titling}

% Create subtitle command for use in maketitle
\providecommand{\subtitle}[1]{
  \posttitle{
    \begin{center}\large#1\end{center}
    }
}

\setlength{\droptitle}{-2em}

  \title{Georeferencing in Q}
    \pretitle{\vspace{\droptitle}\centering\huge}
  \posttitle{\par}
    \author{Sam Levin}
    \preauthor{\centering\large\emph}
  \postauthor{\par}
      \predate{\centering\large\emph}
  \postdate{\par}
    \date{5/6/2019}


\begin{document}
\maketitle

\begin{enumerate}
\def\labelenumi{\arabic{enumi}.}
\item
  Load in the TIFF from the first sampling in the Layers panel. This
  will allow you to find reference points to snap the new TIFF to.
\item
  Open plugin manager (\emph{Plugins} -\textgreater{} \emph{Manage and
  Install Plugins\ldots{}})
\item
  Start typing ``Georeferencer GDAL'' into the search bar
\item
  Once you see ``Georeferencer GDAL'' appear in the menu, check the box
  next to it. Close the dialogue box
\item
  Select the \emph{Raster} dropdown and open the Georeferencer plugin
\item
  Click the \emph{Add Raster} button (top left corner) and select the
  TIFF from the current year
\item
  In the settings menu, select \emph{Thin Plate Splines} from the
  Transformation dropdown.

  \begin{itemize}
  \tightlist
  \item
    Also experimenting with Helmert transformations.
  \end{itemize}
\item
  Make a name for the output layer. This should be the site name, the
  year, and the georeferencing method used (e.g.~TPS, Helmert, etc).
\item
  Find pixels that are clearly in both TIFFs, and mark it the
  georeferencer.
\item
  A popup dialogue will appear, select the Find in Map Canvas option,
  and click the corresponding point on the previous year's TIFF.
\item
  Repeat 9 and 10 as many times as possible.
\item
  Once you have created all of the georeferencing points, save them by
  going to \emph{File} -\textgreater{} \emph{Save GCPs as\ldots{}}.

  \begin{itemize}
  \item
    There is now a GCPs folder in the \emph{PhD\_Processed\_Data/}
    subfolder. If the country/site does not already have its own folder,
    create one.
  \item
    The naming convention for GCPs is \emph{site\_Year1\_Year2.points}.
    Saving these means others can check our work or extend the work on
    their own.
  \end{itemize}
\end{enumerate}


\end{document}
